\documentclass[a4paper,10pt]{book}
\usepackage[utf8]{inputenc}
\usepackage[T2A]{fontenc}
\usepackage[warn]{mathtext}
\usepackage[russian]{babel}
\usepackage{amsmath, amsfonts, amssymb}
\let\vec\mathbf

\begin{document}
% Раздел 2
% Лекция 13
\part{Раздел II}
\chapter{Лекция 13}
\section*{Понятие о токе}
В предыдущем разделе рассматривались неподвижные электрические заряды и создаваемое ими электрическое поле. Перейдём 
к изучению движущихся зарядов. Электрическим током называют направленное перемещение заряженных частиц или движение 
заряженных тел. Ток, образуемый движением заряженных микрочастиц в твердых, жидких или газообразных телах под действием
электрического поля называют \emph{током проводимости}. Заряженные частицы, движение которых образует электрический ток,
получили название \emph{носителей тока}.

В металлах ток создается движние свободных электронов. В жидкостях носителями тока служат положительные и отрицательные 
ионы (ионный ток). В газах ток создается движением положительных и отрицательных ионов, а также электронов. Носителями 
тока в полупроводниках являются электроны и ``дырки'' (свободные места, на которые могут переходить электроны), частично 
подобные положительные положительным зарядам.

Кроме тока проводимости различают еще \emph{ток в вакуума}, например поток электронов в электронной лампе, телевизионной 
трубке. 

Движение заряженных тел (макроскопических) называют \emph{конвекционным (переносным) током}.

При движении любой заряженной частицы или тела в окружающем пространстве образуется магнитное поле. Поэтому основным свойством 
всякого тока - проводимости, в вакууме и конвекционного - является образование магнитного поля (магнитное действие тока). Прох
ождение тока в твердых\footnote{За исключением сверхпроводников}, жидких и газообразных телах сопровождается их нагреванием в 
результате частичного превращения упорядоченного действия тока в хаотическое (тепловое действие тока).

Во всех случаях ионного тока наблюдается перенос вещества и некоторые химические процессы (химическое действие тока).
\section{Сила тока}
Силой тока \emph{(i, I)} называют скалярную величину равную отношению количества электричества \emph{dq}, проходящего через некоторую
поверхность, ко времени прохождения \emph{dt}. Обычно этой поверхностью служит поперечное сечение проводника.
\begin{equation}\label{iless}
 i = \frac{\mathrm{d}q}{\mathrm{d}t}
\end{equation}
Формула \ref{iless} пригодна и для меняющегося и для постоянного тока. Если же сила тока неизменна во времени, то величина её \emph{I}
определится как отношение \emph{q} к \emph{t}.
\begin{equation}\label{igreat}
 I = \frac{q}{t}
\end{equation}
Иногда вместо термина ``сила тока'' говорят просто ``ток''.

Единицей силы тока служит ампер (\emph{a}). Определение этой основной единицы Международной системы приведено в разделе ``Электромагнетизм''
(лекция 34).

Из формулы \ref{igreat} можно определить производную единицу заряда в СИ
\begin{equation}
 1 \text{к} = 1 \text{а} \cdot 1 \text{сек} \nonumber
\end{equation}
Один кулон - это заряд, который проходит через сечение проводинка за 1 \emph{сек} при силе тока в 1 \emph{а}. При этом через сечение проводника
пройдёт 
\begin{equation}
 N = \frac{1}{e} = \frac{1 \text{к}}{1,6 \cdot 10^-19 \text{к}} = 6,25 \cdot 10^18 \nonumber
\end{equation}
элементарных зарядов.

Формула \ref{igreat} используется в системе СГС для определения единицы силы тока
\begin{equation}\label{sgsamp}
  1\text{СГС}_I = \frac{1\text{СГС}_q}{1 \text{cек}} = \frac{1}{3 \cdot 10^9} \text{а}
\end{equation}
\section{Плотность тока}
В учении о токе важную роль играет векторная величина плотность тока \textbf{j}, численно равная силе тока, отнесенное к единице площади
поперечного сечения проводника с током.
\begin{equation}
 j = \frac{I}{S}
\end{equation}
Плотность тока измеряется в $\text{а/м}^2$ (внесистемная единица $1\frac{\text{а}}{\text{мм}^2} = 10^6\frac{\text{а}}{\text{м}^2}$).

Если рассмотреть проводник с переменным сечением или проводящую среду, то плотность тока \emph{j} будет величиной переменной
\begin{equation}\label{dencity}
 j = \frac{\mathbf{d}I}{\mathbf{d}S_n}
\end{equation}
где $dS_n$ - перпендикулярный к направлению плотности тока элемент площади.

Тогда
\begin{equation}\label{inti}
 I = \int\limits_{s}j\mathbf{d}S_n
\end{equation}
т.е. сила тока является потоком от вектора плотности тока через заданную поверхность (см. лекцию 2).
\section{Закон Ома}
В 1827 г. немецкий учитель физики Ом установил опытным путём пропорциональность между напряжением \emph{U}, приложенным 
к участку цепи, и током, созданным в нём. Отношение напряжения (разности потенциалов) к силе тока в данном участке цепи есть величина
постоянная называемая сопротивлением участка
\begin{equation}\label{resist}
 \frac{U}{I} = R
\end{equation}
Как известно, сопротивление проводника \emph{R} постоянного сечения связанно с его длиной \emph{L}, площадью поперечного сечения \emph{S}
и удельным сопротивлением $\rho$ следующим соотношением:
\begin{equation}\label{rls}
 R = \rho\frac{L}{S} = \frac{1}{\gamma}\frac{L}{S}
\end{equation}
Величина, обратная сопротивлению, - $G = \frac{1}{R}$ называется проводимостью, а $\gamma = \frac{1}{\rho}$ - удельной проводимостью
(электропроводимостью).

Из формулы \ref{resist} устанавливается единица для измерения сопротивления в СИ
\begin{equation}
 1 \text{ом} = \frac{1\text{в}}{1\text{a}},
\end{equation}
т.е. 1 \emph{ом} является сопротивлением проводника, в котором идет ток в 1 \emph{а} при напряжении 1\emph{в} между его концами.

Удельное сопротивление $\rho$ численно равно сопротивлению проводника из данного материала, длиной в 1 \emph{м} и поперечным сечением
в 1 $\text{м}^2$ (в СИ); измеряется величина $\rho$ в $\frac{\text{ом} \cdot \text{м}^2}{\text{м}} = \text{ом} \cdot \text{м}$.
Внесистемная единица $1 \text{ом} \cdot \text{см} = 10^-2 \text{ом} \cdot \text{м}$.

Как известно из электростатики, напряженность однородного поля численно равна падению потенциала на единицу расстояния
\begin{equation}\label{Ell}
 E = \frac{U}{L}
\end{equation}
Используя формулы \ref{igreat} \ref{dencity} \ref{resist} \ref{rls} и \ref{Ell}, получаем следующее выражение для плотности тока:
\begin{equation}\label{longf}
 j = \frac{I}{S} = \frac{U}{RS} = \frac{U}{\rho\frac{L}{S}S} = \frac{1}{\rho}\frac{U}{L} = \gamma E
\end{equation}
где \emph{E} - напряженность электрического поля, созданного в проводнике.

Так как \textbf{E} - величина векторная, то
\begin{equation}\label{vecj}
 \vec{j} = \gamma\vec{E}
\end{equation}
т.е. вектор \textbf{j} совпадает по направлению с вектором напряженности электрического поля.

Формула \ref{vecj} выражает закон Ома в \emph{дифференциальной форме}. Величина силы тока получается в общем случае путем интегрирования
плотности тока по площади [см. формулу \ref{inti}]. Плотность тока оказывается прямопопорциональной напряженности поля, созданного в проводнике.
\section{Работа и мощность тока. Закон Джоуля-Ленца в дифференциальной форме}
При прохождении тока по проводнику совершается работа по перенесению заряда q между точками с разностью потенциалов \emph{U}. Эта работа 
затрачивается на нагревание проводника
\begin{equation}\label{work}
 A = qU = IUt = I^2Rt = \frac{U^2}{R}t = Q
\end{equation}
В Международной системе единиц работа \emph{A} и количество тепла \emph{Q} измеряются в джоулях. Поэтому формула \ref{work} для работы 
тока одновременно выражает закон Джоуля-Ленца о тепловом действии тока. 

Заметим, что при \emph{последовательном} соединении сила тока \emph{I} во всех сечениях одинакова и поэтому, согласно формуле $Q = I^2Rt$,
количество тепла , выделяющееся на каком-либо участке цепи, пропорционально его сопротивлению $R$.

При \emph{параллельном} соединении во всех ветвях имеется общее напряжение U и количество тепла
\begin{equation}
 Q = \frac{U^2}{R}t, \nonumber
\end{equation}
выделяющееся в какой-либо ветви, обратно пропорционально ее сопротивлению, т.е. пропорционально ее проводимости.

Подсчитаем количество тепла, выделяемое за единицу времени в единице объема $V$ проводника, т.е. найдём 
\begin{equation}
 Q_1 = \frac{Q}{tV} = \frac{Q}{tSL}. \nonumber
\end{equation}
Используя соотношения \ref{rls}, \ref{Ell} и \ref{work}, получаем:
\begin{equation}\label{warm}
 Q_1 = \frac{U^2}{RSL} = \frac{E^2L^2}{\frac{1}{\gamma}\frac{L}{S}SL} = \gamma E^2.
\end{equation}
\emph{Колличество тепла, выделяемое в единице объема проводника за единицу времени, пропорционально квадрату напряженности электрического 
поля созданного в проводнике.}

Соотношение \ref{warm} называется законом Джоуля-Ленца в дифференциальной форме, так как определяет количество тепла, выделяемое в 1 сек
в единице объема.

В общем случае количнство тепла, выделяемое в 1 сек во всем объеме,
\begin{equation}\label{warmVol}
 Q_V = \int\limits_{(V)}Q_1\mathbf{d}V
\end{equation}
Рассматривая формулы \ref{vecj} и \ref{warm}, замечаем, какую важную роль играет величина напряженности электрического поля $E$, созданного
источником в проводнике. Величина $E$ определяет плотность тока в цепи и количество выделяемого током тепла.
\chapter{Лекция 14}
\section*{Замкнутая цепь с источником тока}
Рассмотрим замкнутую цепь электрического тока, составленную из источника тока и внешнего сопротивления $R$ \ref{img1}. Во внешней части цепи
заряды движутся под действием электростатических сил от большего потенциала к меньшему (путь $1R2$). Для того чтобы заставить заряды
двигаться внутри источника тока против направления электростатического поля (от 2 к 1 через внутренней сопротивление $r$), необходимо наличие
сил неэлектрического происхождения, так называемых сторонних сил.

Величину, определяемую работой, которую совершают сторонние силы при перемещении единичного положительного заряда по всей замкнутой цепи,
называют электродвижущей силой (э. д. с.) $\mathcal{E}$ в этой цепи (не путать с напряженностью электрического поля $\mathbf{E}$ и ее 
числовым значением $E$).
\begin{equation}\label{eds}
 \mathcal{E} = \frac{A_\text{ст}}{q_0}
\end{equation}
% here img1
Этой работой измеряется электростатическая энергия, которая получается в источнике тока за счет иных, неэлектрических форм энергии.

Из определения ЭДС видно, что это величина, аналогичная разности потенциалов (напряжению), имещая ту же размерность и измеряемая в тех же
единицах - вольтах.

Падение напряжения во внешней части цепи $U_1$ измеряется работой, совершаемой при перемещении единичного положительного заряда через 
внешнее сопротивление $R$
\begin{equation}\label{u1}
 U_1 = \frac{A_1}{q_0}
\end{equation}
При прохождении тока по всей замкнутой цепи внутри источника также затрачивается энергия на продвижение зарядов и имеет место внутреннее
падение напряжения $U_2$ эта энергия превращается в джоулево тепло внутри источника.
\begin{equation}\label{u2}
 U_2 = \frac{A_2}{q_0}
\end{equation}
Учитывая, что все три величины - $\mathcal{E}, U_1, \text{и} U_2$ измеряются работой по перемещению единичного положительного заряда и 
применяя закон сохранения энергии, получим
\begin{equation}\label{Aex}
 A_\text{ст} = A_1 + A_2
\end{equation}
\begin{equation}\label{Aex/q0}
 \frac{A_\text{ст}}{q_0} = \frac{A_1}{q_0} + \frac{A_2}{q_0} \text{ и } \mathcal{E} = U_1 + U_2.
\end{equation}
Эдс равна сумме падений напряжения во внешней и внутренней частях замкнутой цепи.

Устройства, в которых сторонние силы совершают работы и происходит превращение какого-либо вида энергии в электрическую, называются генераторами,
или источниками эдс (реже - источниками тока)ю

Подавляющая часть электрической энергии совершается за счет совершения механической работы в машинах, где используется явление электромагнитной
индукции (генераторы на стационарных и передвижных электростанциях). В термоэлементах (термопарах) происходит превращение тепловой энергии 
в электрическую. В гальванических элементах и аккумуляторах химическая энергия преобразуется в электрическую. Некоторые виды фотоэлементов
позволяют получать эдс за счет лучистой энергии.

Таким образом, следует различать электромеханические, тепловые, химические и оптические генераторы, или источники эдс. В настоящее время
разрабатываются новые генераторы - магнитогидродинаические, где тепловая энергия нагретого ионизированного газа или дыма непосредственно 
превращается в электрическую.

Получение различных радиоактивных изотопов позволило создать маломощные генераторы длительного действия; в них эдс получается за счет непрерывного
выбрасывания электронов радиоактивными ядрами.
\section{Закон Ома для всей цепи}
Используя закон Ома для участка цепи, можно заменить в формуле \ref{Aex/q0} $U_1$ и $U_2$ следующими выражениями:
\begin{equation}
 U_1 = IR,\text{ }U_2 = Ir.\nonumber
\end{equation}
Тогда соотношение \ref{Aex/q0} превратится в закон Ома для всей цепи: 
\begin{equation}\label{full_Ohm}
 \frac{\mathcal{E}}{I} = R + r
\end{equation}
Отношение эдс, действующей в замкнутой цепи, к силе тока есть величина постоянная, равная полному сопротивлению цепи или 
\begin{equation}\label{full_I}
 I = \frac{\mathcal{E}}{R + r}.
\end{equation}
Максимальный ток $I_\text{к.з.}$, как следует из закона Ома, получится если внешнее сопротивление R = 0, т.е. имеет место короткое замыкание
\begin{equation}\label{I_shcut}
 I_\text{к.з.} = \frac{\mathcal{E}}{r}
\end{equation}
Записав формулу \ref{full_Ohm} в виде
\begin{equation}
 \mathcal{E} = U_1 + Ir\nonumber
\end{equation}
можно определить эдс как величину, численно равную напряжению на зажимах источника при $I = 0$, т.е. при разомкнутой внешней цепи.
Этим пользуются для практического измерения эдс. Точное измерение эдс производится компенсационным способом (известным из лабораторной 
работы), при котором ток от источника не потребляется. Приближенно эдс измеряется вольтметром с большим сопротивлением.
\section{Полная и полезная мощноть. КПД источника}
Мощность, выделяемая во внешней цепи (нагрузке $R$), называется полезной
\begin{equation}
 P_1 = U_1I = I^2R\nonumber
\end{equation}
Внутри источника тока расходуется мощность 
\begin{equation}
 P_2 = I^2r\nonumber
\end{equation}
Полная мощность замкнутой цепи
\begin{equation}\label{full_P}
 P_n = I^2R + I^2r = IU_1 + IU_2 = I\mathcal{E}.
\end{equation}
Найдем условие, при котором полезная мощность будет максимальной. Для этого выразим полезную мощность как 
\begin{equation}
 P_1 = P_n - P_2 = I\mathcal{E} - I^2r\nonumber
\end{equation}
и приравняем нулю производную
\begin{equation}
 \frac{\mathbf{d}P_1}{\mathbf{d}I} = \mathcal{E} - 2Ir = 0,\nonumber
\end{equation}
откуда
\begin{equation}\label{profit_I}
 I = \frac{\mathcal{E}}{2r}.
\end{equation}
Сравнивая эту формулу с законом Ома \ref{full_I}, видим, что максимальна мощность во внешней цепи выделяется при равенстве внешнего и 
внутреннего сопротивления:
\begin{equation}\label{exeqin}
 R = r.
\end{equation}
Коэффициентом полезного действия $\eta$ источника эдс называется отношение полезной мощности к полной
\begin{equation}\label{nu}
 \eta = \frac{P_1}{P_n} = \frac{I^2R}{I^2(R + r)} = \frac{U_1}{\mathcal{E}} = \frac{R}{R + r}
\end{equation}
Наибольший кпд, равный единице, получится при равенстве нулю внутреннего сопротивления источника. Исходя из этого стараются делать внутреннее
сопротивление источника эдс (обмоток механического генератора, электролита и пластин аккумулятора) минимальным.




% Лекция 19 конец
\chapter{Лекция 19}
\section*{...}
На рис. 41 показана схема, с помощью которой можно наблюдать явление Пельтье. Если переключатель К находится в положении 1, то через термобатарею пропускается постоянный ток от источника. При этом благодаря явлению Пельтье одни спаи будут нагреваться, а другие охлаждаться. Затем с помощью переключателя К термобатарею подключают к гальванометру (положение 2). По цепи проходит ток, вызванный термо-э.д.с. (последняя возникает из-за различной температуры спаев).


% Лекция 20
\chapter{Лекция 20}
\section*{Электрический ток в вакууме. Термоэлектронная эмиссия}
Электроны могут не только переходить из металла в металл; при наличии у них энергии, превышающей работу выхода они могут вырываться из поверхности металла в вакуум. Испускание электронов за счёт кинетической энергии их теплового движения называют термоэлектронной эмиссией; выбивание электронов их металлов квантами света получило название фотоэффекта; вырывание электронов вследствие удара электронов и ионов о поверхность металла называется вторичной электронной эмиссией а «вытягивание» электронов из катода сильным электрическим полем — автоэлектронной эмиссией. Термоэлектронная эмиссия была открыта американским изобретателем Эдисоном и английским физиком Ричардсоном, которые обнаружили, что раскалённые металлы испускают электроны. Впоследствии испускание электронов при сильном нагреве было обнаружено и у других тел.

Если считать, что электроны образуют в металле электронный газ, то средняя кинетическая энергия электронов будет
\begin{equation}
\frac{mu_T^2}{2} = \frac{3}{2}kT\nonumber
\end{equation}
Чтобы она была достаточна для совершения работы выхода $A$, должно быть
\begin{equation}\label{lecture20_1}
\frac{3}{2}kT \geq A
\end{equation}
Отсюда
\begin{equation}
T \geq \frac{2A}{3k}\nonumber
\end{equation}
Полагая среднюю работу выхода равной $2 \text{ эв}$ = $2*1,6 * 10^{-19} \text{ дж}$, получим
\begin{equation}
T \geq \frac{2*2*1.6*10^{-19}}{3*1.38*10^{-23}}\approx 15 000^{\circ}\text{ К}\nonumber
\end{equation}
Опыт показывает, что уже при температурах в 1000–2500$^{\circ}$К металлы испускают значительное количество электронов. Это обусловлено тем, что электроны в металлах определённым образом распределены по энергии и часть их имеет значительно большие энергии, чем $W_\text{ср}$. Эти электроны могут вылетать из металла уже при 100$0^{\circ}$.

\section{Электрический ток в вакууме}
Ток в вакууме может быть создан при наличии источников заряженных частиц, например, раскалённой металлической нити — катода, металлической пластинки, облучаемой светом и испускающей электроны благодаря фотоэффекту, и др.

На рис. 42 показана схема вакуумной электронной лампы с двумя электродами — катодом и анодом. Здесь термоэлектронная эмиссия вызывается нагреванием катода электрическим током от источника напряжения накала $U_\text{н}$. Вылетевшие из катода электроны образуют электронное облако, электрическое поле которого налагается на поле, созданное электродами. Поле электронного облака будет препятствовать удалению вновь вылетающих электронов из катода. Поэтому часть их возвращается в металл. В конце концов создаётся динамическое равновесие между числом вылетающих из металла и возвращающихся в него электронов, напоминающее процесс, происходящий при наличии насыщенного пара над жидкостью.

Приложим теперь напряжение $U_\text{а}$ (называемое анодным) между анодом и катодом. Созданное в ваккуме электрическое поле заставит электроны двигаться к положительно заряженному аноду и через электронную лампу пойдёт ток $I_\text{a}$. По мере увеличения $U_\text{a}$ всё большее число электронов из объъёмного заряда будет доходить до анода и величина $I_\text{a}$ будет расти. Наконец, наступит такое соотношение, при котором все электроны, испускаемые катодом, имеющим температуру $T$, долетают до анода, и объёмный заряд уничтожается. Дальнейшее увеличение анодного напряжения $U_\text{а}$ не влияет на силу анодного тока. Ток, сила которого не увеличивается с ростом напряжения, называется \emph{током насыщения} $I_\text{нас}$. По силе тока насыщения судят о числе электронов $N$, ежесекундно покидающих катод. Очевидно, что


\begin{equation}
N = \frac{I_\text{нас}}{e}.\nonumber
\end{equation}

\section{Электрический ток в вакууме}
Графически изображённая зависимость между силой тока и напряжением в данном приборе, называется вольтамперной характеристикой.

Изменяя анодное напряжение $U_\text{а}$ с помощью реостата $R_\text{н}$ (см. рис. 42) и измеряя это напряжение вольтметром, а анодный ток миллиамперметром, можно получить вольтамперную характеристику тока в вакууме (рис. 43). Вначале (пока лишь часть электронов достигает анода) анодный ток растёт благодаря отсасыванию электронов из объёмного заряда полем анода. Как показали американский инженер Лэнгмюр и московский физик Богуславский, зависимость $I_\text{a}$ от $U_\text{а}$ в начале характеристики выражется «законом трёх вторых»

\begin{equation}\label{zakon3/2}
I_\text{а} = \alpha U_\text{а}^{\frac{3}{2}}.
\end{equation}

Коэффициент пропорциональности $\alpha$ зависит от геометрических размеров, формы и взаимного расположения электродов и вычисляется или измеряется отдельно для каждой конструкции ламп.

Полукубическая парабола, уравнение которой дано формулой \ref{zakon3/2}, описывает рост анодного тока в функции от анодного напряжения на участке $Om_1$, $Om_2$.

Наибольшее значение анодного тока, получаемое при определенной температуре катода, и есть ток насыщения. Его величина зависит только от температуры, материала и величины поверхности катода. При повышении температуры ток насыщения быстро растёт (по показательному экспоненциальному закону).

Нижняя кривая на рис. 43 снята при температуре катода $T_1$, верхняя — при более низкой температуре $T_2$.

На рис. 44,\textit{а} показана опытная зависимость между током насыщения и температурой катода.

Теоретическая формула, связывающая ток насыщения, температуру катода и работу выхода электронов по энергиям согласно Ферми (см. лекцию 17) и называется формулой Ричардсона. Обычно формула Ричардсона приводится для плотности тока насыщения:
\begin{equation}
j_\text{нас} = \frac{I_\text{нас}}{S_\text{кат}} = BT^{2}e^{-\frac{A}{kT}}.
\end{equation}
Здесь $S_\text{кат}$ — площадь поверхности катода, $T$ — температура катода, $^{\circ}\text{ К}$, $A$ — работы выхода электронов из данного металла, $B$ — константа, равная для чистых металлов $1,2*10^{6} \text{а/м}^{2}*\text{град}$.

Увеличение тока насызения может быть достигнуто, во-первых, уменьшением работы выхода (путём покрытия катода очень тонкий слоем тория, окиси бария) и, во-вторых, — увеличением температуры. Оба эти фактора сильно влияют на $I_\text{нас}$ благодаря наличию в формуле Ричардсона экспоненциального множителя. Так, повышение температуры вольфрамовой нити накала от $2000$ до $3000^{\circ}\text{ К}$, т. е. в полтора раза, увеличивает $I_\text{нас}$ в несколько миллионов раз.

На рис. 44, б изображена зависимость плотности тока насыщения (в логарифмическом масштабе) от величины, обратной температуре катода $(\frac{1}{T})$.


% Лекция 21
\chapter{Лекция 21}
\section*{Эмиссия электронов. Электронные приборы. Диод}
\section{Электронные приборы}
Двухэлектродная лампа — диод — представляет собой стеклянный или металлический баллон, в котором создан вакуум и впаяны два электрода — катод и анод. Диод используется в качестве выпрямителя и в качестве детектора электромагнитных колебаний в многочисленных радиотехнических и электротехнических устройствах. Выпрямляющий диод (кенотрон) пропускает ток, когда потенциал холодного электрода (анода) выше, чем горячего электрода (катода); при изменении знака разности потенциалов электрическое поле отбрасывает электроны назад, к горячему электроду, и ток через лампу не идёт (запирающее напряжение).

Если к диоду приложено переменное, например синусоидальное, напряжение, то он срезает отрицательные «полуволны» и создаёт импульсы тока (рис. 45,\textit{а} и 45,\textit{б}).

Для двухполупериодного выпрямителя используется двойной диод — лампа с одним катодом и двумя анодами (рис. 46). При этом в нагрузочном сопротивлении $R$ протекает пульсирующий ток (рис. 45,\textit{в}). 

Катод присоединяется к средней точке вторичной обмотки (точки \textit{1} и \textit{2}). Ток идёт через каждый анод только в течение одного полупериода. Пусть, например, полуволна переменного напряжения направлена сначала от \textit{2} к \textit{1}; ток пойдёт от точки \textit{1}, через первый анод $A_1$, на катод, через нагрузочное сопротивление $R$, в точку \textit{3} и вернётся в точку \textit{1}. В следующем полупериоде направление тока изменится — он будет протекать по обмотке трансформатора от \textit{1} к \textit{2}, на второй анод $A_2$, через катод, сопротивление $R$, в точки \textit{3} и \textit{2}.

В оба полупериода ток через сопротивление $R$ протекает в одном и том же направлении. Обмотка $T_\text{н}$ служит для питания цепи накала катода. Обычно пульсацию тока сглаживают путём зарядки и разрядки конденсатора, включаемого параллельно сопротивлению $R$ (на схеме не показан).

Для управления потоком электронов в лампе служит третий электрод — сетка, располагаемый между катодом и анодом. Сетка расположеная ближе к катоду, чем анод, и её потенциал сильнее влияет на анодный ток, чем потенциал анода.

Действительно, для создания у катода одинаковой напряжённости электрического поля $E$ на более далёкий анод надо подавать более высокое напряжение, чем на сетку. Вспомнив, что напряжённость электрического поля равна градиенту потенциала\begin{equation}
E = \frac{\Delta U}{\Delta l}\nonumber
\end{equation} и обозначив расстояние от катода до сетки $\Delta l_c$, а до анода $\Delta l_a$, полчаем, что при одинаковом E\begin{equation}
\Delta U_c = -E\Delta l_c,\nonumber
\end{equation} а \begin{equation}
\Delta U_a = -E\Delta l_a.\nonumber
\end{equation}

Так как $\Delta l_a>\Delta l_c$, то по абсолютному значению $\Delta U_a>\Delta U_c$. Величина плотности анодного тока $j_a$ в лампе зависит от напряжённости поля, созданного потенциалами анода и сетки; рост $E$ вызывает увеличение плотности тока.

Лампа с сеткой — триод служит для усиления малых напряжений. На рис. 47 изображена схема включения триода (источник тока для накала катода обычно не показывается).

Сеточная характеристика триода, т. е. зависимость анодного тока $I_a$ от напряжения на сетке $U_c$, показана на рис. 48. Здесь изображены две сеточные характеристики, снятые при различных напряжениях на аноде $(U_a2>U_a1)$.

При отсутствии напряжения на сетке анодный ток создаётся анодным напряжением. Для того чтобы $I_a=0$, необходимо приложить к сетке отрицательное запирающее напряжение $U_\text{зап}$. Обычно на сетку подаётся некоторое отрицательное напряжение, называемое начальным смещением $U_{c_0}$.

Если на сетку подавать малые изменения напряжения $\Delta U_c$, то они вызовут изменения анодного тока на величину $\Delta l_a$; при этом на большом нагрузочном сопротивлении произойдёт падение напряжения $\Delta U_a=R_\text{н} \Delta I_a$, большее, чем $\Delta U_c$. В этом состоит усилительное действие лампы.

Триод принято характеризовать тремя параметрами: крутизной характеристики $S$, коэффициентом усиления $\mu$ и внутренним сопротивлением $R_i$.

Рассмотрим треугольник $abc$ на прямолинейном участке сеточной характеристики (рис. 48).

Отношение приращения анодного тока $\Delta I_a$ к вызвавшему его изменению сеточного напряжения $\Delta U_c$ при неизменном напряжении на аноде (например, $U_{a_2}$), называют крутизной характеристики
\begin{equation}
S = (\frac{\Delta I_a}{\Delta U_c})_{U_a=const}.
\end{equation}

Отношение приращения анодного напряжения $\Delta U_a$ к изменению сеточного напряжения $\Delta U_c$, при котором анодный ток остаётся постоянным, называется статическим коэффициентом усиления триода $\mu$ \begin{equation}\label{mu_21.2}
\mu = - (\frac{\Delta U_a}{\Delta U_c})_{I_a=const}.
\end{equation}

Для сохренения анодного тока неизменным надо, например, увеличить анодное напряжение $(\Delta U_a > 0)$) и уменьшить потенциал сетки $(\Delta U_c < 0)$. Поэтому в формуле \ref{mu_21.2} стоит знак минус, т. е сама величина $\mu$ положительна. Можно дать следующее простое определение важнейшего параметра триода — коэффициента усиления $\mu$: величина $\mu$ показывает, во сколько раз сеточное напряжение сильнее влияет на анодный ток, чем анодное.

Внутренним сопротивлением лампы $R_i$ называют отношение приращения анодного напряжения $\Delta U_a$ к вызванному им приросту анодного тока $\Delta I_a$ при постоянном сеточном напряжении\begin{equation}\label{21.3}
R_i = (\frac{\Delta U_a}{\Delta I_a})_{U_c=const},
\end{equation}
причём $\Delta U_a = U_{a_2}-U_{a_1}$, т. е. рассматриваются две сеточные характеристики при различных анодных напряжениях и постоянном $U_c$. Легко показать, что\begin{equation}\label{21.4}
\frac{S R_i}{\mu} = 1.
\end{equation}

Действительно, \begin{equation}
\frac{\frac{\Delta I_a}{\Delta U_c}*\frac{\Delta U_a}{\Delta I_a}}{\frac{\Delta U_a}{\Delta U_c}} = 1.\nonumber
\end{equation}

Соотношение \ref{21.4} называется уравнением Баркгаузена.

Для ламп с одной сеткой величина $\mu \approx 10 \div 15$. Для большего усиления напряжение, снятое с сопротивления $R_\text{н}$, подаётся на сетку следующей лампы (второй каскад усиления), и т. д. Кроме этого, в лампу можно ввести вторую, третью сетки, повысив этим $\mu$ до $100 \div 1000$.

\section{Умножители. Вторичная электронная эмиссия}

При ударе электронов и ионов о металлы и полупроводники происходит выбивание электронов из твёрдых веществ, т. е. имеет мето так называемая вторичная электронная эмиссия. Коэффициентом вторичной электронной эмиссии $\delta$ назовём отношение числа выбитых вторичных электронов $N_2$ к числу первичных падающих $N_1$.

\begin{equation}
\delta = \frac{N_2}{N_1}.
\end{equation}

Этот коэффициент для металлов колеблется от 1,25 (Ni) до 1,78 (Pt), а для полупроводников достигает значения 10 и более.

На описанном явлении основано действие так называемых умножителей, разработанных Кубецким, Векшинским и другими учёными.

Рассмотрим фотоэлектронный умножитель (ФЭУ), рис. 49. При освещении первого катода слабым пучком света из него благодаря фотоэффекту вырываются электроны. Проследим за движением одного из них. Ускоряясь в поле с разностью потенциалов $U$ между первым и вторым катодом, этот электрон выбивает из второго катода $\delta$ электронов. Пролетая от второго катода к третьему, электроны снова ускоряются той же разностью потенциалов и выбивают уже $\delta*\delta=\delta^{2}$ электронов. При наличии $n$ электродов в умножителе число вылетевших их последнего катода электронов больше числа вырванных их первого в $\delta^{n-1}$ раз.

ФЭУ гашли широкое применение для усиления слабых фототоков, например для счёта частиц в ядерной физике.

\section{Автоэлектронная эмиссия}

Если создать в вакууме сильное электрическое поле, прикладывая между катодом и анодом напряжение поряжка нескольких тысяч вольт, то из холодного катода, выполненного в форме острия, начнут вырываться электроны.

Явление вырывания электронов из холодного катода сильным электрическим полем называется автоэлектронной эмиссией. Наибольший градиент потенциала\begin{equation}
E = \frac{dU}{dr}
\end{equation} получается как раз вблизи катода. При радиусе кривизны острия $r=10^{-6} \textit{ м} = 1 \textit{ мкм} $ и разности потенциалов в трубке, равной 1000 $\textit{в}$, напряжённость поля у катода достигает громадной величины — $10^(8)\textit{ в/м} $.

При автоэлектронной эмиссии работа выхода в значительной мере преодолевается за счёт работы, совершаемой над электронами сильным внешним электрическим полем.

Явление автоэлектронной эмиссии играет исключительно важную роль при образовании и поддержании электрической дуги (см. лекцию 24).


\end{document}